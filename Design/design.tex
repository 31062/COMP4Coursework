\chapter{Design}

\section{Overall System Design}

\subsection{Short description of the main parts of the system}

\subsection{System flowcharts showing an overview of the complete system}

\section{User Interface Designs}

\section{Program Structure}

\subsection{Top-down design structure charts}

\subsection{Algorithms in pseudo-code for each data transformation process}

\subsection{Object Diagrams}

\subsection{Class Definitions}

\section{Prototyping}

\section{Definition of Data Requirements}

\subsection{Identification of all data input items}

\subsection{Identification of all data output items}

\subsection{Explanation of how data output items are generated}

\subsection{Data Dictionary}
\begin{center}
\begin{tabular}{|l|l|l|l|l|l|}
    \hline
    \textbf{Name} & \textbf{Data Type} & \textbf{Length} & \textbf{Validation} & \textbf{Example Data} & \textbf{Comment} \\ \hline
	stock ID & int & 2 bytes & 4 digits, must exsist & 3845 & unique to each stock item\\ \hline
	user ID & int & 2 bytes & 2 digits, must exsist & 04 & unique to each member of staff\\ \hline
	producer & string & 12 bytes & 4-12 chariters & taddingtons & name of product producer\\ \hline
	supplier & string & 14 bytes & 4-14 chariters & milton brewery & name of product supplier \\ \hline
	retail price & float & 2 bytes & int or decimal & 3.10 & price of a product for one base unit\\ \hline
	unit retail & int & 4 bytes & int 1-3 digits & 750 & base unit the product is sold in\\ \hline
	stockname & string & 14 bytes & string 4-14 & dead pony club & name of a product\\ \hline
	unit purchase & int & 3 bytes & 144 & int 1-144 & 72 & amount of the product\\ \hline
	unit cost & int & 3 bytes & int 0-150 & 89 & cost of unit purchase\\ \hline
	stock type & string & 8 bytes & 5-8 chariters & real ale & type of stock\\ \hline
	 
\end{tabular}
\label{tab:range_examples}
\end{center}

\subsection{Identification of appropriate storage media}

  


\section{Database Design}

my data base will have 10 tables; users, stock, stock type, real ale, snacks, spirits, unit retail, unit purchase, producer, supplier, permissions.
user (\underline{user ID}, user name, \emph{permissions}, postcode, city, street, house no., landline, mobile no)
stock (\underline{stock ID}, stock name, \emph{stock type}, \emph{unit retail}, retail price, amount used, amount remaining, sell by date, alcohol percentage)
stock type(\underline{stock type ID}, stock type)
producers(\underline{producer ID}, producer)
supplier(\underline{supplier ID}, supplier)
permissions(\underline{permission ID}, permission)
real ale(\underline{real ale ID}, ale name, alcohol percentage, \emph{unit purchase}, \emph{unit retail}, \emph{producer})
snacks(\underline{snack ID}, snack name, \emph{unit purchase}, \emph{unit retail}, \emph{producer})
spirits(\underline{spirts ID}, spirit name, \emph{unit purchase}, \emph{unit retail}, \emph{producer})
larger(\underline{larger ID}, larger name)
unit retail(\underline{unit retail ID}, unit retail)
unit purchase(\underline{unit purchase ID}, unit purchase)

\subsection{Normalisation}

\subsubsection{unf}





\subsubsection{ER Diagrams}

\subsubsection{Entity Descriptions}

userID; is used to identify each differnt user of the system so that the manegers can look through who did what on the system, we use userID to identify each user as multiple can have the same name.

user name; is used to print the name of the user on screen as well as some documents such as recets

\subsubsection{1NF to 3NF}

\section{Security and Integrity of the System and Data}

\subsection{Security and Integrity of Data}

\subsection{System Security}

\section{Validation}

\section{Testing}

\begin{landscape}
\subsection{Outline Plan}

\begin{center}
    \begin{tabular}{|p{2cm}|p{5cm}|p{5cm}|p{4cm}|}
        \hline
        \textbf{Test Series} & \textbf{Purpose of Test Series} & \textbf{Testing Strategy} & \textbf{Strategy Rationale}\\ \hline
        Example & Example & Example & Example \\ \hline
    \end{tabular}
\end{center}

\subsection{Detailed Plan}

\begin{center}
    \begin{longtable}{|p{1.5cm}|p{2.5cm}|p{2.5cm}|p{2cm}|p{2cm}|p{2cm}|p{2cm}|p{2cm}|}
        \hline
        \textbf{Test Series} & \textbf{Purpose of Test} & \textbf{Test Description} & \textbf{Test Data} & \textbf{Test Data Type (Normal/ Erroneous/ Boundary)} & \textbf{Expected Result} & \textbf{Actual Result} & \textbf{Evidence}\\ \hline
        Example & Example & Example & Example & Example & Example & Example & Example \\ \hline
    \end{longtable}
\end{center}
\end{landscape}