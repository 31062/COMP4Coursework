\chapter{System Maintenance}

\section{Environment}

\subsection{Software}
In creating this system have have lots of different software to preform different tasks: Python 3; is a loosely typed 3rd generation programing language. IDLE; is where you physically write your code for python, IDLE is used to write python script as well as run a python script, including displaying errors. SQLite 3; is part of the standard library of python and is used to create and use databases within python. datetime; datetime is part of the standard library of python and is used so programmers can use the time and date in there programs. SQLite inspector; this is a program that displays data with in a database it can also execute queries

\subsection{Usage Explanation}

\subsubsection{python 3}
The reasons I choose to write my system in python firstly is because python is the programing language i am most familiar with however there are a one or two other programs i could have written in. python is also a loosely typed language making it easier to use and understand. python is also easy to installed on any machine making it a good choice when I have write parts of this system from three different machines all with different hardware and software installed on them. python also has a standard library which can be imported into any python script for the programmer to use.

\subsubsection{IDLE}
IDLE is the program in which one write their python code, IDLE is able to save written code as well as run it and display errors. IDLE has an easy to understand bug report format with a feature to jump you to the line of code which is incorrect as well as giving useful information about the type of error and what Modules the incorrect line is part of. this makes IDLE the ideal environment in which to program my system. 

\subsubsection{SQLite 3}
I used SQLite 3 in order to create and access databases from a python script. SQLite 3 has simple syntax making it easy to understand and use, this also makes it possible to create much more complicated queries to access, order and sort data.

\subsubsection{datetime}
i used the datetime library to time stamp certain data being entered into the database such as DeliveryTimeDate and the time of each transaction.

\subsection{Features Used}

\subsubsection{python}
In python I used the import feature in order to import functions from other python files into the main file, i also used the return feature in order to pass parameters between the function.

\subsubsection{}


\subsubsection{}


\subsubsection{}


\subsubsection{}


\section{System Overview}

%use as many subsections as necessary for the system components
\subsection{System Component}

\section{Code Structure}

%use as many subsections as necessary for the code sections
\subsection{Particular Code Section}
%the code below can be uncommented and used to get a code section from a particular file
\begin{comment}
\begin{figure}[H]
    \pythonfile[firstline=5,lastline=10]{./tex/function_programs/print_function.py}
    \caption{The print() function} \label{fig:print_function}
\end{figure}
\end{comment}

\section{Variable Listing}

\section{System Evidence}

\subsection{User Interface}

\subsection{ER Diagram}

\subsection{Database Table Views}

\subsection{Database SQL}

\subsection{SQL Queries}

\section{Testing}

\subsection{Summary of Results}

\subsection{Known Issues}

\section{Code Explanations}

\subsection{Difficult Sections}

\subsection{Self-created Algorithms}

\section{Settings}

\section{Acknowledgements}

\section{Code Listing}
\begin{landscape}
%include as many subsections as you have modules
\subsection{Module 1}
%the code below can be uncommented and used to get a code section from a particular file
\begin{comment}
\pythonfile[firstline=5]{./tex/function_programs/print_function.py}
\end{comment}
\end{landscape}
