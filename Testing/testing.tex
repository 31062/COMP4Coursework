\chapter{Testing}

\section{Test Plan}

\begin{landscape}
\subsection{Original Outline Plan}

\begin{center}
    \begin{tabular}{|p{2cm}|p{5cm}|p{5cm}|p{4cm}|}
        \hline
        \textbf{Test Series} & \textbf{Purpose of Test Series} & \textbf{Testing Strategy} & \textbf{Strategy Rationale}\\  \hline
        1.1 & to test the main menu functions proberly  & black bock testing & the use will only enter one valur into this menu and this part of the system only directs the user to a sub menu there is no real precessing thus it is best to test with what output the system gives \\ \hline
1.2 &to test the "insert new line into database" menu & white box testing & i will use white box testing as there are multiple ways the user can get to the end of this sub menu via any of the 12 options.\\ \hline
1.3 & to test the "edit currenlty exsisting line in database" menu & white box testing & i will use white box testing 12 differnt initial options which then can have a munber of secondery options. this plart of the program also gives no prompt that the database has been changed so black box testing it not an option. \\ \hline
1.4 & to test "make transation" menu & black box testing & this part of the system can have many differnt roots to get through it but these option are only minor such as a differnt user or stock item. \\ \hline
1.5 & to test "delete line from database" menu & black box testing & this part of the system is very easy to tell if it is working properly because it has a lot of output steaments therefore black box testing is ok to use \\ \hline
    \end{tabular}
\end{center}

\subsection{Changes to Outline Plan}

\subsection{Original Detailed Plan}

\begin{center}
    \begin{longtable}{|p{1.5cm}|p{2.5cm}|p{2.5cm}|p{2cm}|p{2cm}|p{2cm}|p{2cm}|p{2cm}|}
        \hline
        \textbf{Test Series} & \textbf{Purpose of Test} & \textbf{Test Description} & \textbf{Test Data} & \textbf{Test Data Type (Normal/ Erroneous/ Boundary)} & \textbf{Expected Result} & \textbf{Actual Result} & \textbf{Evidence}\\ \hline
       1.1.1 & to test the funtionalty of the "create new database" option is correct when no database exsists & i will load the program i will then enter 1 into the system to create a new database. & 1 & normal & the program will create a database and state this on screen &system created a new database called pub_stock.db & Example \\ \hline
1.1.2 & to test the funtionalty of the "create new database" option is correct when there is an exsisting database  &  i will load the program i will then enter 1 into the system to create a new database & 1 & normal &the program will crash due to a database called pub_stock already exsists & the system opens the current database and trys to create a new table with the same name of an exsisting table this then crashes the program & \\ \hline
1.1.2 & to test "insert new line in database"functionalty is correct & i will enter the value 2 into the main menu screen

    \end{longtable}
\end{center}

\subsection{Changes to Detailed Plan}

\begin{center}
    \begin{longtable}{|p{1.5cm}|p{2.5cm}|p{2.5cm}|p{2cm}|p{2cm}|p{2cm}|p{2cm}|p{2cm}|}
        \hline
        \textbf{Test Series} & \textbf{Purpose of Test} & \textbf{Test Description} & \textbf{Test Data} & \textbf{Test Data Type (Normal/ Erroneous/ Boundary)} & \textbf{Expected Result} & \textbf{Actual Result} & \textbf{Evidence}\\ \hline
        Example & Example & Example & Example & Example & Example & Example & Example \\ \hline
    \end{longtable}
\end{center}

\section{Test Data}

\subsection{Original Test Data}

\subsection{Changes to Test Data}

\section{Annotated Samples}

\subsection{Actual Results}

\subsection{Evidence}

\end{landscape}

\section{Evaluation}

\subsection{Approach to Testing}

\subsection{Problems Encountered}

\subsection{Strengths of Testing}

\subsection{Weaknesses of Testing}

\subsection{Reliability of Application}

\subsection{Robustness of Application}